
\chapter{Introduction}

In a embedded devices development context, platform simulation is referring to a method of simulating 
the behavior of an entire platform, or only a few selected components, in order to test and evaluate
the functionality and performance of the system before it is deployed. This can be done using specialized
software tools and techniques that simulate the hardware, software, and other components of the platform,
allowing developers to run and test their code in a virtual (simulated) environment before it is implemented
onto the physical hardware.

Platform simulation became an indispensable tool in the modern-day embedded system development. Such technology
provides each engineer with an dedicated, deterministic and reproducible system for development, testing and debugging.
This approach allows the developers to use advanced debugging software to perform a wide range of tests to ensure that
the embedded system is functioning correctly and meeting the desired performance and reliability requirements. 
This allows developers to detect, identify and fix bugs early in the development process, saving time and resources.

The advantages of the platform simulation also reach beyond the software development phase, and well into the further
phases of the product lifespan. One such advantage is the ability to easily integrate these systems into existing
continuous integration (CI) systems that check for regressions with each software/hardware iteration.
This is because the entire platform is contained within a simulation layer, making integration much simpler than it
would be with hardware platforms, which would require a hardware/software integration layer. An another example of
the benefit brought by emulation is an ability to create software, without having an hardware platform ready.
This is an especially important in the post COVID-19 pandemic times, where the supply chains have been massively
disrupted, with lead time on some parts, especially for the most advanced elements, rising as much as four times,
from about four weeks, in the begging of the year 2020, to over twenty weeks at the end of 2021
\cite{Covid19-AUTOMOTIVE} \cite{Covid19-LEAD-TIME} \cite{Covid19-LEAD-TIME-BLOOMBERG}. Hardware emulation allows for
the parallelization of the software and hardware development, enabling developers to work on both aspects
simultaneously and reducing the overall time and effort required to bring a product to market.

\section{Motivation and goals of the thesis}

A large part of the recognized platform and/or core architecture emulators use opcode translation to run guest code
on the host platform. The most prevalent and widely used open source tools in this category include Renode \cite{Renode}
and QEMU \cite{Qemu}.%
\footnote{The Renode's CPU translation library is partly based on QEMU}
Due to major optimization this method of the simulation yields great results performance-wise, on the other hand
such heavy efficacy improvements come with a price of greatly complicating the codebase, making it difficult
to maintain and integrate new functionality. Another downside of the translation based approach is the need
to execute the guest code, as this requires complicated just in time (\textit{JIT}) recompilers.

