
\chapter{Introduction}

In a embedded devices development context, platform simulation is referring to a method of simulating 
the behavior of an entire platform, or only a few selected components, in order to test and evaluate
the functionality and performance of the system before it is deployed. This can be done using specialized
software tools and techniques that simulate the hardware, software, and other components of the platform,
allowing developers to run and test their code in a virtual (simulated) environment before it is implemented
onto the physical hardware.

Platform simulation became an indispensable tool in the modern-day embedded system development. Such technology
allows the developers to perform a wide range of tests, such as unit testing, stress testing, and smoke and sanity
testing, to ensure that the embedded system is functioning correctly and meeting the desired performance and 
reliability requirements. This approach allows developers to detect, identify and fix bugs early in the development
process, saving time and resources.

% TODO:
% The first sentence sucks
The use of a fully virtual approach to software development for embedded systems has many advantages.
One such advantage is the ability to easily integrate these systems into existing continuous integration (CI) systems.
This is because the entire platform is contained within a simulation layer, making integration much simpler than it
would be with hardware platforms, which would require a hardware/software integration layer. An another example of
the benefit brought by emulation is an ability to create software, without having an hardware platform ready.
This is an especially important in the post COVID-19 pandemic times, where the supply chains have been massively
disrupted, with lead times on some parts, especially for the most advanced elements, such as central processing units
(CPU) and Graphics processing units (GPUs) rose nearly four times, from about four weeks, in the begging of the year 2020,
to over twenty weeks at the end of 2021 \cite{Covid19-AUTOMOTIVE} \cite{Covid19-LEAD-TIME} 
\cite{Covid19-LEAD-TIME-BLOOMBERG}.

