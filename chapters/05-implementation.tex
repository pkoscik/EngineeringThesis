
\chapter{Integrating Dromajo with Renode framework}

This chapter aims to introduce the reader to general implementation concepts of the integration of the Dromajo simulator with the Renode Framework.
To provide a reader with a better understanding of the integration process, this chapter also covers
the basics of Dromajo and Renode, the structure of simulated devices, and the basics of CPU simulation
implementation in the framework.

In order to integrate Dromajo with the Renode simulation framework, the following steps must be completed:
\begin{itemize}
    \item{creating an external simulator API for Dromajo,}
    \item{creating a DromajoCPU interface in Renode,}
    \item{binding Dromajo's C library to Renode C\# managed code.}
\end{itemize}

\noindent
The process of implementation is outlined in two distinct stages. The first part describes the implementation of an
external simulator API in Dromajo. The second part focuses on the Renode side and covers the details of creating a
DromajoCPU instance and binding C\# managed code to C++ library.

\section{Dromajo}

Dromajo is an Esperanto Technologies RISC-V (RV64GC) emulator designed for RTL co-simulation.%
\footnote{RTL (register-transfer level) co-simulation is a technique used in hardware design and verification to
simulate the behavior of a digital circuit design (CPU) at the RTL level.}
%Hey, this sentence is a copy-paste. NO GO
Dromajo's semantic model is based on Fabrice Bellard's RISCVEMU (later renamed TinyEMU), but extensively verified,
bug-fixed, and enhanced to take it to ISA 2.3/priv 1.11 \cite{Dromajo}.
\footnote{Fabrice Bellard is also the original author of TCG and QEMU.}
It has been released under Chips Alliance
\cite{ChipsAlliance} under an Apache 2.0 License.

Dromajo at the core is either an RTL co-simulation engine or a whole machine emulator, both implementing an
% ideally you'd not write Chapter 4, but Chapter \ref{chap:interpret}. But I'm being picky.
Interpretation CPU emulator, which inner workings have been described in the Chapter 4. Dromajo does not provide any API
% My wording here, about intro work, is deliberate. You remember our discussion about the thesis being about research, not implementation
for integration with other simulators, which was resolved as a part of the introductory work to this thesis. The majority of the
codebase is written in C, but due to a few C++ dependent functionality (for example \texttt{LiveCache}), the project is
compiled as a C++ project, using \texttt{Cmake} build engine.
% Please let me know what is this livecache

%it's usually a bad idea to have manual pagebreaks...
\pagebreak

\subsection{External Simulator API for Dromajo}

The Dromajo CPU simulator operates on an internal emulator state (\texttt{struct RISCVCPUState}), therefore each
instance of the library must implement its own, global-scoped \texttt{*env} state. Since this approach is not considered
% I do not understand. Whose approach was this? Did we enforce global env, did we need it? I don't get it.
a good practice in most cases, the existing methods were not compatible with such approach. This was solved by
creating \texttt{wrappers} for already existing methods, that passed the global state as a parameter to existing
functions. To implement this approach with as few changes to an already existing source code as possible, there was a need to
extract the essential functionality to an independent compile unit.%
\footnote{A compilation unit, often referred to as translation unit, is a final representation of the source code,
from which the object file is generated.}

\vspace{10px}
\noindent
List of exposed API functions (the \texttt{dromajo\_} prefix has been omitted in all function names):
\begin{multicols}{2}
    \begin{itemize}
        \item{\textbf{init\_cpu}\\Initializes the \texttt{*env} state, sets a \texttt{hartid} }
        \item{\textbf{map\_range}\\Registers a RAM page in the emulator}
        \item{\textbf{virt\_to\_phys}\\Translates guest to host RAM address}
        %is this supposed to be bit_bit? I don't think so
        \item{\textbf{set\_mip\_bit\_bit\_state}\\Sets or resets selected MIP CSR flag}
        \item{\textbf{execute}\\Interprets a given number of instructions} % I added "given", becuse it looked like "interprets what the number means"
        \item{\textbf{get\_power\_down\_flag}\\Returns the CPU power down flag}
        \item{\textbf{get\_pc}\\Returns the current program counter}
        \item{\textbf{set\_pc}\\Sets the program counter to a requested value}
        \item{\textbf{get\_hart\_id}\\Returns the CPU \texttt{hartid}}
        \item{\textbf{set\_hart\_id}\\Sets the \texttt{hartid} to a requested value}
        \item{\textbf{get\_register}\\Returns a selected CPU register}
        \item{\textbf{set\_register}\\Sets a CPU register to a requested value}
    \end{itemize}
\end{multicols}

\noindent
The listing below presents a sample implementation:

% wait, is it ACTUALLY bit_bit? That's fishy
\begin{lstlisting}[
    style=lstC,
    label={lst:dromajo-api-state},
    caption={Examples of External Simulator API function wrapper.}
    ]
RISCVCPUState *env;

void dromajo_set_mip_bit_bit_state(uint32_t number, BOOL value) {
    uint32_t mask = 1 << number;

    //The riscv_cpu_* functions are a part of existing Dromajo code
    value ? riscv_cpu_set_mip(env, mask) : riscv_cpu_reset_mip(env, mask);
}

uint32_t dromajo_get_power_down_flag() {
    return riscv_cpu_get_power_down(env);
}
...
\end{lstlisting}

\noindent
This interface is good enough for basic usage, but it does not enable the external simulator to control the CPU
execution flow precisely. This issue is discussed on in the next section.

% pagebreaks are fishy
\pagebreak

% either interface, or api
\subsection{External Simulator API for Dromajo}

To further extend the functionality of the API, the \textit{External Simulator Interface} has been also added
to more comprehensively integrate the CPU emulator into an external simulator. The main motivation for this
feature is the ability to add breakpoints and to provide an organized interface to add callbacks to an external
simulator. This functionality implements a \texttt{struct ExternalSimulator} embedded, as a member, into the
\texttt{RISCVCPUState}.

% duuuuuuude!
\vspace{10px}
\noindent
List of exposed interface API functions (the \texttt{external\_sim\_} prefix has been omitted in all function names):
\begin{multicols}{2}
    \begin{itemize}
        \item{\textbf{init}\\Initializes \texttt{ExternalSimulator} with provided external simulator callbacks}
        \item{\textbf{get\_execution\_stopped\_flag}\\Returns the exit status of the execution:
        \texttt{\{OK,Breakpoint,WFI\}}}
        \item{\textbf{get\_executed\_instructions}\\Returns the ammount of executed insn. since the last call}
        \item{\textbf{check\_breakpoint}\\Checks if breakpoint exists at current PC}
        \item{\textbf{set\_ignore\_bp\_flag}\\Tells the emulator to ignore the next breakpoint}
        \item{\textbf{set\_any\_bp\_flag}\\Tells the emulator to ignore every breakpoint}
        \item{\textbf{add\_breakpoint}\\Adds a breakpoint at a given PC}
        \item{\textbf{remove\_breakpoint}\\Removes a breakpoint at a given PC}
    \end{itemize}
\end{multicols}

\noindent
The listing below presents a sample of the implementation:

\begin{lstlisting}[
    style=lstC,
    label={lst:dromajo-interface},
    caption={Examples of External Simulator API function wrapper.}
    ]
uint64_t external_sim_get_executed_instructions() {
    uint64_t result = sim->instruction_count_value;
    sim->instruction_count_value = 0;
    return result;
}

BOOL external_sim_check_breakpoint(target_ulong pc) {
    Breakpoints* bps = sim->breakpoints;

    for (size_t i = 0; i < bps->size; i++) {
        Breakpoint bp = bps->entries[i];
        if (likely(bp.pc == pc && bp.active)) {
            return TRUE;
        }
    }
    return FALSE;
}
\end{lstlisting}

\noindent
Such an interface provides a tightly coupled external simulator\ $\leftrightarrow$\ CPU emulator relation. Not only does
it organizes callbacks, but also provides necessary functionality to implement extended flow control, used for example
for debugging using GDB.

\pagebreak

\subsection{External Simulator Register API}

To standardize register numbering convention%
\footnote{In the case of RISC-V, the ABI (Application Binary Interface)
already does that, but only for general purpose registers. The CSRs (Control and Status Registers) and Floating Point
register numbering is implementation specific.}%
, external API implements a \texttt{DROMAJO\_REGISTERS} enumeration, that aims to provide a stable representation of
% Is it based on Renode numbering? If so, we can say that explicitly.
all implemented registers.

\begin{lstlisting}[
    style=lstC,
    label={lst:dromajo-register-interface},
    caption={Dromajo API register convention enumeration.}
    ]
typedef enum {
    ZERO     = 0,
    RA       = 1,
    SP       = 2,
    GP       = 3,
    TP       = 4,
    FP       = 8,
    PC       = 32,
    SSTATUS  = 321,
    SIE      = 325,
    STVEC    = 326,
    SSCRATCH = 385,
    SEPC     = 386,
    ...
} DROMAJO_REGISTERS
\end{lstlisting}

\subsection{Building Dromajo as a library}

To build Dromajo as an external simulator library, the \texttt{CMakeLists.txt} has been extended with an additional
target:

\begin{lstlisting}[
    label={lst:dromajo-cmakelists},
    caption={Dromajo \texttt{CMakeLists.txt}}
    ]
install(TARGETS dromajo_cosim DESTINATION .)
add_library(dromajo_external_simulator SHARED
        src/external_simulator_api.cpp
        src/external_simulator_interface.cpp
        src/external_simulator_registers.cpp
        ...
        )
target_compile_definitions(dromajo_external_simulator PUBLIC EXTERNAL_SIMULATOR)
target_compile_definitions(dromajo_external_simulator PUBLIC EXTERNAL_SIMULATOR_TIME_CSR)
\end{lstlisting}

\noindent
The last two target compile definitions are used in the preprocessing step of the compilation and are used to enable
or disable certain functionalities at the compile time. Building a project with this target will result in an output
of a \texttt{dromajo\_external\_simulator.so}, a shared library that can be used in any external simulator.

\pagebreak

\section{Renode Framework}
The Renode Framework is Antmicro's open-source, permissively licensed (MIT), embedded device simulation framework.
Being mostly written in C\#, the framework provides an organized, easily expandable codebase, rich in
object-oriented principles. It can be compiled using both \texttt{mono} and \texttt{.NET} build systems for
all major operating systems.

% wide range, for me, is many hw targets - hence I added levels
The framework provides users with the ability to emulate a wide range of microcontroller-enabled solutions on several levels:
% You need to rework grammar here. SoC is used as in a sentence, and Device/Multi-node are different
\begin{itemize}
    \item{\textbf{System on Chip} emulation allows the user to assemble virtual System-on-Chips from building blocks,
    including Cortex-M, RISC-V, and other architectures, as well as a variety of peripherals, communication buses and interfaces.}
    %
    \item{\textbf{Device}, by adding additional elements, such as buttons, LEDs, communication modules, etc. users can
    % digital twin is a good buzzword to use
    emulate entire devices, such as development boards or digital twins of actual products}
    %
    \item{\textbf{Multi-node systems} connecting two or more devices using a simulated communication protocol opens
    up a possibility of simulating and testing the deployment of systems at scale.}
\end{itemize}

\noindent
This granularity provides the users and developers with a way to easily test and create software for the desired
development platforms, create functional and performance tests, and easily debug the internal workings of the platform.
All of this functionality is confined to a single piece of software, eliminating the need to create costly, or in some
cases unaffordable physical testing rigs, while providing the freedom to copy, distribute and scale as needed.

With Industry 4.0 well on the rise, the importance of rapid development cycles, throughout testing and verification
on each level of the infrastructure (especially at scale), and debugging abilities is paramount.

\section*{Source code structure}
% "submodule" is very technical
The framework source code \cite{Renode} is divided into three modules:

\begin{itemize}
    \item{\textbf{Renode} is the main project repository, it contains the general project resources, build
    scripts, required libraries, additional tools, and plugins for external integration.}
    %
    \item{\textbf{Infrastructure} is the main emulation layer for peripherals, and an interface for CPU emulation
    libraries. This submodule also provides internal interfaces and frameworks used for modeling devices}
    %
    \item{\textbf{Tlib}, a shorthand for \textit{Translation Libraries}, is an implementation of a Translation CPU
    simulator}
\end{itemize}

\noindent
The work done in this thesis mainly focuses on the Infrastructure module, as this is the part that already implements
the Translation CPU emulator interface.

\section*{CPU emulation in Renode}

The CPU is simulated using the beforementioned \textit{Translation Libraries}, a Translation CPU simulator
% Qemu was a long time ago. We definitely don't write that we're forked, it's not true
based on the TCG concept, supporting Cortex-M, Cortex-A, RISC-V, PPC, SPARC, Xtensa, and i386.\\
Translation Libraries is written in C, compiled as a shared library (\texttt{.so}), and then binded to Renode using
Renode's C\# utility called \texttt{NativeBinder}.

\pagebreak

\subsection{Implementing DromajoCPU class in Renode}

The Renode Framework provides a CPU stub class, \texttt{ExternalCPU.cs}, that can be utilized to implement other
emulators into the framework. This has been used as a template and expanded upon when integrating Dromajo.
Here we will only discuss the most important functionality and interfaces.

\vspace{10px}
\noindent
The \texttt{DromajoCPU} class implements the following interfaces and classes:
\begin{multicols}{2}
    \begin{itemize}
        \item{\textbf{BaseCPU}\\Implements the basic CPU functionality in Renode}
        % saying "GPIO" is technically wrong in this context. Our interface name sux
        \item{\textbf{IGPIOReciver}\\Exposes the peripheral to signal input, used for IRQs}
        \item{\textbf{ITimeSink}\\Represents an object that is aware of virtual time flow}
        \item{\textbf{ICPUWithMappedMemory}\\Implements pointer-based memory accesses}
        \item{\textbf{ICPUWithRegisters}\\Enables Renode to access CPU registers}
        \item{\textbf{ICPUWithHooks}\\Adds hooks interface to the CPU}
        \item{\textbf{ICpuSupportingGdb}\\Exposes the CPU to the GDB stub}
    \end{itemize}
\end{multicols}

\noindent
The following sections describe each of the interfaces in more detail.

\subsection*{GPIO Receiver}
This interface allows the CPU to be connected in a Renode Platform description (\texttt{.resc}) to other devices.
% well, its not cpu -> plic, but plic -> renoce
On an example of RISC-V, the CPU receives signals from a Platform Level Interrupt Controller \textit{(PLIC)} and
Core Level Interrupt Controller \textit{(CLINT)}. Both of these devices handle interrupt requests and provide similar
functionality to ARM's Nested Vector Interrupt Controller.

\begin{lstlisting}[
    label={lst:renode-repl},
    caption={Renode \texttt{.repl} file with \textit{DromajoCPU}}
    ]
U74_1: CPU.DromajoCPU @ sysbus
    cpuType: "dromajo"
    hartId: 0
    ...

clint: IRQControllers.CoreLevelInterruptor  @ sysbus 0x02000000
    [0, 1] -> U74_1@[3, 7]
    ...

plic: IRQControllers.PlatformLevelInterruptController @ sysbus 0x0c000000
    [0, 1] -> U74_1@[11, 9]
    ...
\end{lstlisting}

\noindent
This interface definies one method, used for handling the IRQs

\begin{lstlisting}[
    style=lstCsharp,
    label={lst:renode-ongpio},
    caption={DromajoCPU \texttt{OnGPIO}}
    ]
public void OnGPIO(int number, bool value)
{
    DromajoSetMipBit((uint)number, value ? 1u : 0u);
}
\end{lstlisting}

\pagebreak

\subsection*{Time Sink}
This interface enables the peripheral to use the \textit{Virtual Time Framework}, an integral part of Renode, used to
separate real and simulated time flow. While this interface only provides one member, \texttt{TimeHandle}, a correct
implementation of this interface and its features has proved to be crucial. Incorrect handling of virtual time resulted
in a non-deterministic and buggy behavior of the guest software.

\subsection*{CPU With Mapped Memory}
To understand this interface, the introduction to the Renode's guest memory handling is necessary. The framework
provides guest with two possible memory access flows, with access method being the main differentiator:

\begin{itemize}
    \item{\textbf{MappedMemory}: a type of native-code accessed memory, initialized and managed by C\# (freed, etc.),
    a region that is \textit{mapped} (via pointers) to the simulator's internal structures.} Using this type of memory
    all of the CPU memory accesses are executed directly in the native (C,C++) code. This approach avoids a slow C\#
    callback upon every access, allowing the simulated CPU to run faster.
    %
    \item{\textbf{IO Access}: a type of managed-code accessed memory, this method uses C\# callbacks from the native
    code to access the memory (accessing the memory via \texttt{SystemBus} methods). The IO access is much simpler,
    at a cost of performance degradation. This method should be only used for accessing peripherals, but
    if desired might be used for accessing regular memory as well.}
\end{itemize}

\noindent
Implementing this interface on a class allows the CPU to utilize MappedMemory implementation, via the \texttt{void MapMemory()} method:

\begin{lstlisting}[
    style=lstCsharp,
    label={lst:renode-mapped-memory},
    caption={DromajoCPU \texttt{MapMemory}}
    ]
public void MapMemory(IMappedSegment segment)
{
    currentMappings.Add(new SegmentMapping(segment));
    DromajoMapRange(offset, size);
}
\end{lstlisting}

\subsection*{CPU With Registers}
% That allow the CPU to expose its registers to the rest of the framework via Renode Register A
This interface implements methods that allow the CPU to use Renode Register API, the beforementioned Dromajo's External
Simulator Register API mapping is used here. This interface is a requirement for \texttt{ICpuSupportingGdb}.

\subsection*{CPU With Hooks}
% I would suggest rewriting this a little. It's not about execution control, it's about reactivity. You don't have to pause at address, you can send an email or run another app if you want
This interface allows the CPU to use methods for advanced execution control, such as \texttt{AddHook, RemoveHook},
allowing Renode to stop the execution of the emulated processor at any address, at any point in time.
This interface is a requirement for \texttt{ICpuSupportingGdb}.

\pagebreak

\subsection*{CPU Supporting GDB}

This interface enables the functionality of the CPU in the Renode GDB stub services. When this interface is correctly
implemented, the processor can be debugged using GDB's Remote Server Protocol.

\subsection{Binding and callbacks}

The last point of the implementation covers the issue of language binding. This is handled by Renode, with the help
of the custom NativeBinder utility. When the DromajoCPU class is created, the NativeBinder object is instantiated with
the Dromajo library path as a parameter. The binding utility then loads the library, parses symbols and attempts to resolve
\textbf{Calls To Managed} (C $\rightarrow$ C\#) and \textbf{Calls To Native} (C\# $\rightarrow$ C).

\subsection*{Resolving calls to native code}

The utility parses all parent class methods marked with an \texttt{[Import]} attribute, finds the correct symbol in the
loaded library, creates a C\# delegate pointing to the library function, and sets the desired method value to the
matching delegate. For example this method will try to bind to C symbol \texttt{dromajo\_set\_mip\_bit\_bit\_state}:

\begin{lstlisting}[
    style=lstCsharp,
    label={lst:renode-calls-to-native},
    caption={An example of call to native code from C\#. \protect\footnotemark}
    ]
[Import(Name = "dromajo_set_mip_bit_bit_state", UseExceptionWrapper = false)]
private ActionUInt32UInt32 DromajoSetMipBit;
\end{lstlisting}

\footnotetext{The \texttt{ExceptionWrapper} is a mechanism used in Translation Libraries to propagate C\# exceptions
trough the native function calls. This implementation does not support exception propagation, so it is explicitly
disabled.}

\subsection*{Resolving calls to managed code}

This functionality is often referred to as \textit{callbacks}. The process of adding a callback is twofold. The first stage
consists of adding of a \textbf{Callback pointer} and a \textbf{Binder function} to the library. The callback pointer
is a raw pointer to a function of the desired callback delegate type. For example:

\begin{lstlisting}[
    style=lstC,
    label={lst:dromajo-calls-to-managed-binding},
    caption={Callback pointer declaration and Binder function definition.}
    ]
// Callback type
typedef uint32_t (*ext_sim_read_byte_fnc)(uint64_t);

// Callback pointer
ext_sim_read_byte_fnc external_sim_read_byte;

// Binder function
void bind_ext_sim_read_byte(ext_sim_read_byte_fnc ptr) {
    external_sim_read_byte = ptr;
}

// Usage
void foo() {
    uint64_t address = 0xDEADBEEF;
    uint32_t value = external_sim_read_byte(address)
}
\end{lstlisting}

\pagebreak

\noindent
The second stage of creating a callback consists of running the \textbf{Native Binder} function, with a \textbf{managed}
delegate pointer as parameter.

\begin{lstlisting}[
    style=lstCsharp,
    label={lst:renode-calls-to-managed},
    caption={An example of delegate to managed code.}
    ]
[Export(Binder = "bind_ext_sim_read_byte", DelegateType = typeof(FuncUInt32UInt64))]
protected uint ReadByteFromBus(ulong offset) // ulong = uint64_t
{
    return machine.SystemBus.ReadByte(offset);
}
\end{lstlisting}

\noindent
% I'd reword this completely, it's kinda sketchy
After Native Binder sets up the callback, the native library is able to call the managed code.